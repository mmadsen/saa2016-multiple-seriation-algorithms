%%%%%%%%%%%%%%%%%%%% author.tex %%%%%%%%%%%%%%%%%%%%%%%%%%%%%%%%%%%
%
% sample root file for your "contribution" to a contributed volume
%
% Use this file as a template for your own input.
%
%%%%%%%%%%%%%%%% Springer %%%%%%%%%%%%%%%%%%%%%%%%%%%%%%%%%%


% RECOMMENDED %%%%%%%%%%%%%%%%%%%%%%%%%%%%%%%%%%%%%%%%%%%%%%%%%%%
\documentclass[graybox,natbib]{svmult}

% choose options for [] as required from the list
% in the Reference Guide
%% The amssymb package provides various useful mathematical symbols
\usepackage{amssymb,amsmath}

\usepackage{mathptmx}       % selects Times Roman as basic font
\usepackage{helvet}         % selects Helvetica as sans-serif font
\usepackage{courier}        % selects Courier as typewriter font
\usepackage{type1cm}        % activate if the above 3 fonts are
                            % not available on your system
%
\usepackage{makeidx}         % allows index generation
\usepackage{graphicx}        % standard LaTeX graphics tool
                             % when including figure files
\usepackage{multicol}        % used for the two-column index
\usepackage[bottom]{footmisc}% places footnotes at page bottom

\usepackage{lipsum}
\usepackage{url}
\usepackage{gitinfo}
\usepackage[section,ruled]{algorithm}
\usepackage{algorithmic}
\usepackage{boxedminipage}
\usepackage[xetex,bookmarks=true,linkcolor=blue,hyperfootnotes=false,breaklinks=true,citecolor=blue,colorlinks=true]{hyperref}
\usepackage{sistyle}
\usepackage{xspace}
\SIthousandsep{,}

% see the list of further useful packages
% in the Reference Guide

\makeindex             % used for the subject index
                       % please use the style svind.ist with
                       % your makeindex program

%%%%%%%%%%%%%%%%%%%%%%%%%%%%%%%%%%%%%%%%%%%%%%%%%%%%%%%%%%%%%%%%%%%%%%%%%%%%%%%%%%%%%%%%%

\begin{document}
\motto{Manuscript version: \gitCommitterDate: \gitAbbrevHash\xspace -- draft for Electronic Symposium, ``Evolutionary Archaeologies: New Approaches, Methods, And Empirical Sufficiency'' at the Society for American Archaeology conference, April 2016}

\title*{Measuring Cultural Relatedness Using Multiple Seriation Ordering Algorithms}
\titlerunning{Cultural Relatedness Through Multiple Seriation Algorithms}
% Use \titlerunning{Short Title} for an abbreviated version of
% your contribution title if the original one is too long
\author{Mark E. Madsen and Carl P. Lipo}
\institute{Mark E. Madsen \at Dept. of Anthropology, University of Washington, Box 353100, Seattle, WA 98195 \email{mark@madsenlab.org} \and Carl P. Lipo \at Environmental Studies Program and Dept. of Anthropology, Binghamton University, 4400 Vestal Parkway East
Binghamton, NY 13902-6000 \email{clipo@binghamton.edu}}

%
% Use the package "url.sty" to avoid
% problems with special characters
% used in your e-mail or web address
%
\maketitle



\abstract*{Seriation is a long-standing archaeological method for relative dating that has proven effective in probing regional-scale patterns of inheritance, social networks, and cultural contact in their full spatiotemporal context. The orderings produced by seriation are produced by the continuity of class distributions and unimodality of class frequencies, properties that are related to social learning and transmission models studied by evolutionary archaeologists. Linking seriation to social learning and transmission enables one to consider ordering principles beyond the classic unimodal curve. Unimodality is a highly visible property that can be used to probe and measure the relationships between assemblages, and it was especially useful when seriation was accomplished with simple algorithms and manual effort. With modern algorithms and computing power, multiple ordering principles can be employed to better understand the spatiotemporal relations between assemblages. Ultimately, the expansion of seriation to additional ordering algorithms allows us an ability to more thoroughly explore underlying models of cultural contact, social networks, and modes of social learning. In this paper, we review our progress to date in extending seriation to multiple ordering algorithms, with examples from Eastern North America and Oceania.}

\abstract{Seriation is a long-standing archaeological method for relative dating that has proven effective in probing regional-scale patterns of inheritance, social networks, and cultural contact in their full spatiotemporal context. The orderings produced by seriation are produced by the continuity of class distributions and unimodality of class frequencies, properties that are related to social learning and transmission models studied by evolutionary archaeologists. Linking seriation to social learning and transmission enables one to consider ordering principles beyond the classic unimodal curve. Unimodality is a highly visible property that can be used to probe and measure the relationships between assemblages, and it was especially useful when seriation was accomplished with simple algorithms and manual effort. With modern algorithms and computing power, multiple ordering principles can be employed to better understand the spatiotemporal relations between assemblages. Ultimately, the expansion of seriation to additional ordering algorithms allows us an ability to more thoroughly explore underlying models of cultural contact, social networks, and modes of social learning. In this paper, we review our progress to date in extending seriation to multiple ordering algorithms, with examples from Eastern North America and Oceania.}



\section{Introduction}\label{introduction}

Lorem ipsum dolor sit amet, consectetur adipiscing elit. Vestibulum
viverra est est. Proin eget tellus metus. Aenean ac tortor pharetra
libero ultricies sagittis. Nulla facilisi. Cras tincidunt interdum
tellus, quis consectetur nunc facilisis nec. Sed fermentum erat a ligula
posuere quis semper risus ullamcorper. Morbi vel tincidunt augue. Nam
dolor ipsum, sagittis quis dignissim eu, pulvinar sed magna. In interdum
magna eu orci facilisis congue. Cras a tellus et lorem sagittis viverra.
Donec risus lectus, mollis at dignissim viverra, dapibus a nulla.
Vivamus porttitor scelerisque turpis, eget lobortis orci auctor eget.
Donec ultricies enim ac augue porttitor convallis. Pellentesque nisl
lorem, consequat a facilisis in, ornare sed lorem. In luctus, elit ac
mattis dapibus, lacus elit varius tortor, vel sollicitudin massa nisl id
massa. Ut sit amet nibh a sem egestas sollicitudin. Vestibulum
scelerisque, dui at tincidunt accumsan, ipsum enim feugiat neque, vel
interdum turpis lectus sed nisi. Nullam ultrices sodales sem, et
placerat nunc euismod eu. Duis leo lacus, semper quis eleifend vitae,
viverra ut nisl. Vestibulum ante ipsum primis in faucibus orci luctus et
ultrices posuere cubilia Curae; Proin rutrum eleifend est, id tempor
velit viverra sed.

\section{The Seriation Method}\label{the-seriation-method}

\begin{equation}\label{eq:axelrod}p(i,j) = \frac{1}{F} \sum_{f=1}^{F} \delta_{\sigma_f(i)\sigma_f(j)}\end{equation}

Lorem ipsum dolor sit amet, consectetur adipiscing elit. Vestibulum
viverra est est. Proin eget tellus metus. Aenean ac tortor pharetra
libero ultricies sagittis. Nulla facilisi. Cras tincidunt interdum
tellus, quis consectetur nunc facilisis nec. Sed fermentum erat a ligula
posuere quis semper risus ullamcorper. Morbi vel tincidunt augue. Nam
dolor ipsum, sagittis quis dignissim eu, pulvinar sed magna. In interdum
magna eu orci facilisis congue. Cras a tellus et lorem sagittis viverra.
Donec risus lectus, mollis at dignissim viverra, dapibus a nulla.
Vivamus porttitor scelerisque turpis, eget lobortis orci auctor eget.
Donec ultricies enim ac augue porttitor convallis. Pellentesque nisl
lorem, consequat a facilisis in, ornare sed lorem. In luctus, elit ac
mattis dapibus, lacus elit varius tortor, vel sollicitudin massa nisl id
massa. Ut sit amet nibh a sem egestas sollicitudin. Vestibulum
scelerisque, dui at tincidunt accumsan, ipsum enim feugiat neque, vel
interdum turpis lectus sed nisi. Nullam ultrices sodales sem, et
placerat nunc euismod eu. Duis leo lacus, semper quis eleifend vitae,
viverra ut nisl. Vestibulum ante ipsum primis in faucibus orci luctus et
ultrices posuere cubilia Curae; Proin rutrum eleifend est, id tempor
velit viverra sed \citep{mayr1959mayr}.

\section{Experiments}\label{experiments}

Lorem ipsum dolor sit amet, consectetur adipiscing elit. Vestibulum
viverra est est. Proin eget tellus metus. Aenean ac tortor pharetra
libero ultricies sagittis. Nulla facilisi. Cras tincidunt interdum
tellus, quis consectetur nunc facilisis nec. Sed fermentum erat a ligula
posuere quis semper risus ullamcorper. Morbi vel tincidunt augue. Nam
dolor ipsum, sagittis quis dignissim eu, pulvinar sed magna. In interdum
magna eu orci facilisis congue. Cras a tellus et lorem sagittis viverra.
Donec risus lectus, mollis at dignissim viverra, dapibus a nulla.
Vivamus porttitor scelerisque turpis, eget lobortis orci auctor eget.
Donec ultricies enim ac augue porttitor convallis. Pellentesque nisl
lorem, consequat a facilisis in, ornare sed lorem. In luctus, elit ac
mattis dapibus, lacus elit varius tortor, vel sollicitudin massa nisl id
massa. Ut sit amet nibh a sem egestas sollicitudin. Vestibulum
scelerisque, dui at tincidunt accumsan, ipsum enim feugiat neque, vel
interdum turpis lectus sed nisi. Nullam ultrices sodales sem, et
placerat nunc euismod eu. Duis leo lacus, semper quis eleifend vitae,
viverra ut nisl. Vestibulum ante ipsum primis in faucibus orci luctus et
ultrices posuere cubilia Curae; Proin rutrum eleifend est, id tempor
velit viverra sed.

\section{Results}\label{results}

Lorem ipsum dolor sit amet, consectetur adipiscing elit. Vestibulum
viverra est est. Proin eget tellus metus. Aenean ac tortor pharetra
libero ultricies sagittis. Nulla facilisi. Cras tincidunt interdum
tellus, quis consectetur nunc facilisis nec. Sed fermentum erat a ligula
posuere quis semper risus ullamcorper. Morbi vel tincidunt augue. Nam
dolor ipsum, sagittis quis dignissim eu, pulvinar sed magna. In interdum
magna eu orci facilisis congue. Cras a tellus et lorem sagittis viverra.
Donec risus lectus, mollis at dignissim viverra, dapibus a nulla.
Vivamus porttitor scelerisque turpis, eget lobortis orci auctor eget.
Donec ultricies enim ac augue porttitor convallis. Pellentesque nisl
lorem, consequat a facilisis in, ornare sed lorem. In luctus, elit ac
mattis dapibus, lacus elit varius tortor, vel sollicitudin massa nisl id
massa. Ut sit amet nibh a sem egestas sollicitudin. Vestibulum
scelerisque, dui at tincidunt accumsan, ipsum enim feugiat neque, vel
interdum turpis lectus sed nisi. Nullam ultrices sodales sem, et
placerat nunc euismod eu. Duis leo lacus, semper quis eleifend vitae,
viverra ut nisl. Vestibulum ante ipsum primis in faucibus orci luctus et
ultrices posuere cubilia Curae; Proin rutrum eleifend est, id tempor
velit viverra sed.

\section{Discussion}\label{discussion}

Lorem ipsum dolor sit amet, consectetur adipiscing elit. Vestibulum
viverra est est. Proin eget tellus metus. Aenean ac tortor pharetra
libero ultricies sagittis. Nulla facilisi. Cras tincidunt interdum
tellus, quis consectetur nunc facilisis nec. Sed fermentum erat a ligula
posuere quis semper risus ullamcorper. Morbi vel tincidunt augue. Nam
dolor ipsum, sagittis quis dignissim eu, pulvinar sed magna. In interdum
magna eu orci facilisis congue. Cras a tellus et lorem sagittis viverra.
Donec risus lectus, mollis at dignissim viverra, dapibus a nulla.
Vivamus porttitor scelerisque turpis, eget lobortis orci auctor eget.
Donec ultricies enim ac augue porttitor convallis. Pellentesque nisl
lorem, consequat a facilisis in, ornare sed lorem. In luctus, elit ac
mattis dapibus, lacus elit varius tortor, vel sollicitudin massa nisl id
massa. Ut sit amet nibh a sem egestas sollicitudin. Vestibulum
scelerisque, dui at tincidunt accumsan, ipsum enim feugiat neque, vel
interdum turpis lectus sed nisi. Nullam ultrices sodales sem, et
placerat nunc euismod eu. Duis leo lacus, semper quis eleifend vitae,
viverra ut nisl. Vestibulum ante ipsum primis in faucibus orci luctus et
ultrices posuere cubilia Curae; Proin rutrum eleifend est, id tempor
velit viverra sed.

\section{Acknowledgements}\label{acknowledgements}

Lorem Ipsum


%% References with bibTeX database:

\bibliographystyle{model2-names}
\bibliography{saa2016-seriation-multiple-approaches}

\end{document}
